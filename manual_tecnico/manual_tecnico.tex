\documentclass[12pt,a4paper]{article}
\usepackage[utf8]{inputenc}
\usepackage[spanish]{babel}
\usepackage{geometry}
\usepackage{graphicx}
\usepackage{fancyhdr}
\usepackage{xcolor}
\usepackage{titlesec}
\usepackage{hyperref}
\usepackage{enumitem}
\usepackage{booktabs}
\usepackage{longtable}
\usepackage{listings}
\usepackage{float}

% Configuración de página
\geometry{
    top=2.5cm,
    bottom=2.5cm,
    left=3cm,
    right=2.5cm
}

% Configuración de header height
\setlength{\headheight}{25pt}

% Configuración de colores
\definecolor{primaryblue}{RGB}{0,102,204}
\definecolor{secondaryblue}{RGB}{51,153,255}
\definecolor{lightgray}{RGB}{245,245,245}

% Configuración de encabezados y pies de página
\pagestyle{fancy}
\fancyhf{}
\fancyhead[L]{\textcolor{primaryblue}{\textbf{Manual Técnico - UberEats Clone}}}
\fancyhead[R]{\textcolor{primaryblue}{\thepage}}
\renewcommand{\headrulewidth}{0.5pt}
\renewcommand{\headrule}{\hbox to\headwidth{\color{primaryblue}\leaders\hrule height \headrulewidth\hfill}}

% Configuración de títulos
\titleformat{\section}
{\Large\bfseries\color{primaryblue}}
{\thesection}{1em}{}

\titleformat{\subsection}
{\large\bfseries\color{secondaryblue}}
{\thesubsection}{1em}{}

% Configuración de enlaces
\hypersetup{
    colorlinks=true,
    linkcolor=primaryblue,
    filecolor=primaryblue,
    urlcolor=primaryblue,
}

% Configuración de código
\lstset{
    backgroundcolor=\color{lightgray},
    basicstyle=\ttfamily\small,
    breaklines=true,
    captionpos=b,
    frame=single,
    numbers=left,
    numberstyle=\tiny\color{gray},
    showstringspaces=false,
    tabsize=2
}

\begin{document}

% Portada
\begin{titlepage}
    \centering
    \vspace*{2cm}
    
    {\Huge\textbf{\textcolor{primaryblue}{MANUAL TÉCNICO}}}
    
    \vspace{1cm}
    
    {\LARGE\textbf{Aplicación UberEats Clone}}
    
    \vspace{1cm}
    
    {\Large Sistema de Gestión de Pedidos y Entregas}
    
    \vspace{2cm}
    
    \begin{figure}[H]
        \centering
        \textcolor{primaryblue}{\rule{8cm}{0.5pt}}
    \end{figure}
    
    \vspace{2cm}
    
    {\large\textbf{Versión:} 1.0}
    
    \vspace{0.5cm}
    
    {\large\textbf{Fecha:} \today}
    
    \vspace{2cm}
    
    {\large\textit{Documentación Técnica Completa del Proyecto}}
    
    \vfill
    
    {\large\textcolor{primaryblue}{\textbf{Tecnologías: Flutter • Firebase • Google Maps • Riverpod}}}
    
\end{titlepage}

% Índice
\tableofcontents
\newpage

% Contenido del manual
\section{Resumen Ejecutivo}

La aplicación UberEats Clone es un sistema completo de gestión de pedidos y entregas que replica las funcionalidades principales de plataformas de delivery como UberEats. El proyecto implementa un ecosistema completo con roles diferenciados para clientes, tiendas y repartidores, incluyendo características avanzadas como análisis en tiempo real, gestión de inventario, sistema de pagos y comunicación integrada.

\section{Objetivo del Proyecto}

\subsection{Objetivo General}
Desarrollar una aplicación móvil multiplataforma que facilite la gestión integral de pedidos de comida, desde la selección de productos hasta la entrega final, proporcionando una experiencia de usuario fluida y herramientas de gestión empresarial robustas.

\subsection{Objetivos Específicos}
\begin{itemize}[itemsep=0.5em]
    \item Implementar un sistema de autenticación seguro con Firebase Auth
    \item Crear interfaces diferenciadas para clientes, tiendas y repartidores
    \item Desarrollar un sistema de pedidos en tiempo real con seguimiento GPS
    \item Integrar un sistema completo de gestión de pagos
    \item Implementar análisis de ventas y métricas para tiendas
    \item Crear un sistema de gestión de inventario automatizado
    \item Desarrollar notificaciones push y sistema de comunicación
    \item Garantizar escalabilidad y mantenibilidad del código
\end{itemize}

\subsection{Parámetros de Evaluación}
\begin{table}[H]
\centering
\begin{tabular}{@{}ll@{}}
\toprule
\textbf{Criterio} & \textbf{Especificación} \\
\midrule
Funcionalidad & Sistema completo de pedidos y entregas \\
Usabilidad & Interfaz intuitiva y responsive \\
Rendimiento & Tiempo de respuesta < 2 segundos \\
Seguridad & Autenticación robusta y datos encriptados \\
Escalabilidad & Arquitectura modular con Riverpod \\
Multiplataforma & Compatible Android e iOS \\
Tiempo Real & Actualizaciones instantáneas vía Firebase \\
\bottomrule
\end{tabular}
\caption{Parámetros de evaluación del proyecto}
\end{table}

\section{Tecnologías y Lenguajes Utilizados}

\subsection{Lenguajes de Programación}
\begin{itemize}[itemsep=0.5em]
    \item \textbf{Dart}: Lenguaje principal para desarrollo Flutter
    \item \textbf{JavaScript}: Para funciones de Firebase Cloud Functions
    \item \textbf{Kotlin/Java}: Para configuraciones nativas Android
    \item \textbf{Swift/Objective-C}: Para configuraciones nativas iOS
\end{itemize}

\subsection{Framework Principal}
\begin{itemize}[itemsep=0.5em]
    \item \textbf{Flutter 3.x}: Framework multiplataforma de Google
    \item \textbf{Dart SDK}: Entorno de desarrollo
    \item \textbf{Material Design}: Sistema de diseño de Google
\end{itemize}

\subsection{Backend y Base de Datos}
\begin{itemize}[itemsep=0.5em]
    \item \textbf{Firebase Firestore}: Base de datos NoSQL en tiempo real
    \item \textbf{Firebase Auth}: Sistema de autenticación
    \item \textbf{Firebase Storage}: Almacenamiento de archivos
    \item \textbf{Firebase Messaging}: Notificaciones push
    \item \textbf{Firebase Functions}: Lógica del servidor
\end{itemize}

\subsection{Gestión de Estado y Arquitectura}
\begin{itemize}[itemsep=0.5em]
    \item \textbf{Riverpod}: Gestión de estado reactiva
    \item \textbf{Provider Pattern}: Arquitectura de inyección de dependencias
    \item \textbf{Repository Pattern}: Abstracción de datos
\end{itemize}

\subsection{APIs y Servicios Externos}
\begin{itemize}[itemsep=0.5em]
    \item \textbf{Google Maps API}: Mapas y geolocalización
    \item \textbf{Places API}: Búsqueda de lugares
    \item \textbf{Geocoding API}: Conversión de coordenadas
    \item \textbf{Firebase Cloud Messaging}: Notificaciones
\end{itemize}

\subsection{Librerías y Dependencias Principales}
\begin{longtable}{@{}lll@{}}
\toprule
\textbf{Librería} & \textbf{Versión} & \textbf{Propósito} \\
\midrule
flutter\_riverpod & \textasciicircum{}2.4.9 & Gestión de estado \\
firebase\_core & \textasciicircum{}2.24.2 & Configuración Firebase \\
cloud\_firestore & \textasciicircum{}4.13.6 & Base de datos \\
firebase\_auth & \textasciicircum{}4.15.3 & Autenticación \\
google\_maps\_flutter & \textasciicircum{}2.5.0 & Mapas integrados \\
geolocator & \textasciicircum{}10.1.0 & Geolocalización \\
image\_picker & \textasciicircum{}1.0.4 & Selección de imágenes \\
fl\_chart & \textasciicircum{}0.65.0 & Gráficos y análisis \\
firebase\_messaging & \textasciicircum{}14.7.10 & Notificaciones push \\
flutter\_local\_notifications & \textasciicircum{}16.3.0 & Notificaciones locales \\
url\_launcher & \textasciicircum{}6.2.2 & Navegación externa \\
permission\_handler & \textasciicircum{}11.1.0 & Permisos del sistema \\
\bottomrule
\caption{Principales dependencias del proyecto}
\end{longtable}

\section{Arquitectura del Sistema}

\subsection{Patrón de Arquitectura}
El proyecto implementa una \textbf{arquitectura en capas} con separación clara de responsabilidades:

\begin{itemize}[itemsep=0.5em]
    \item \textbf{Capa de Presentación}: Screens y Widgets
    \item \textbf{Capa de Lógica de Negocio}: Providers y Notifiers
    \item \textbf{Capa de Datos}: Services y Repositories
    \item \textbf{Capa de Modelos}: Data Models y DTOs
\end{itemize}

\subsection{Estructura de Directorios}
\begin{lstlisting}[language=bash, caption=Estructura del proyecto]
lib/
|-- models/                    # Modelos de datos
|   |-- payment_model.dart
|   |-- analytics_model.dart
|   |-- inventory_model.dart
|   |-- notification_model.dart
|   |-- chat_model.dart
|   |-- order_model.dart
|-- services/                  # Servicios y APIs
|   |-- payment_service.dart
|   |-- notification_service.dart
|   |-- location_service.dart
|   |-- firebase_service.dart
|-- providers/                 # Gestion de estado
|   |-- payment_provider.dart
|   |-- analytics_provider.dart
|   |-- inventory_provider.dart
|   |-- auth_provider.dart
|-- screens/                   # Pantallas de la aplicacion
|   |-- customer/
|   |-- store/
|   |-- deliverer/
|-- widgets/                   # Componentes reutilizables
|-- utils/                     # Utilidades y helpers
\end{lstlisting}

\section{Funcionalidades Implementadas}

\subsection{Sistema de Autenticación}
\begin{itemize}[itemsep=0.5em]
    \item Registro y login con email/contraseña
    \item Autenticación con Google Sign-In
    \item Gestión de sesiones persistentes
    \item Roles diferenciados (cliente, tienda, repartidor)
    \item Recuperación de contraseñas
\end{itemize}

\subsection{Gestión de Pedidos}
\begin{itemize}[itemsep=0.5em]
    \item Catálogo de productos con filtros
    \item Carrito de compras interactivo
    \item Sistema de reordenar pedidos anteriores
    \item Seguimiento en tiempo real de pedidos
    \item Estados de pedido automatizados
    \item Historial completo de pedidos
\end{itemize}

\subsection{Sistema de Pagos}
\begin{itemize}[itemsep=0.5em]
    \item Múltiples métodos de pago (tarjeta, efectivo, digital)
    \item Validación de tarjetas con algoritmo de Luhn
    \item Cálculo automático de propinas (0\%, 10\%, 15\%, 20\%)
    \item Procesamiento seguro de transacciones
    \item Historial de pagos y reembolsos
    \item Integración preparada para gateways reales
\end{itemize}

\subsection{Análisis y Métricas (Dashboard)}
\begin{itemize}[itemsep=0.5em]
    \item Métricas de ventas en tiempo real
    \item Gráficos de ingresos con fl\_chart
    \item Análisis de productos más vendidos
    \item Identificación de horas pico
    \item Tasas de finalización de pedidos
    \item Reportes por períodos personalizables
\end{itemize}

\subsection{Gestión de Inventario}
\begin{itemize}[itemsep=0.5em]
    \item Seguimiento de stock en tiempo real
    \item Alertas de inventario bajo
    \item Historial de movimientos de inventario
    \item Gestión de proveedores
    \item Costos y precios automatizados
    \item Reportes de rotación de productos
\end{itemize}

\subsection{Sistema de Notificaciones}
\begin{itemize}[itemsep=0.5em]
    \item Notificaciones push con Firebase Messaging
    \item Notificaciones locales por categorías
    \item Configuración de preferencias por usuario
    \item Notificaciones de pedidos, chat, inventario
    \item Horarios de silencio configurables
    \item Badges y contadores de notificaciones
\end{itemize}

\subsection{Comunicación y Chat}
\begin{itemize}[itemsep=0.5em]
    \item Sistema de chat tienda-cliente
    \item Integración con WhatsApp
    \item Mensajes pre-configurados
    \item Estados de mensaje (enviado, entregado, leído)
    \item Soporte para texto, imágenes y ubicación
    \item Notificaciones de nuevos mensajes
\end{itemize}

\subsection{Seguimiento y Entrega}
\begin{itemize}[itemsep=0.5em]
    \item Seguimiento GPS en tiempo real
    \item Mapas interactivos con Google Maps
    \item Rutas optimizadas para repartidores
    \item Estimación de tiempos de entrega
    \item Notificaciones de proximidad
    \item Confirmación de entrega con ubicación
\end{itemize}

\section{Flujo de Lógica del Negocio}

\subsection{Flujo Principal del Cliente}
\begin{enumerate}[label=\arabic*.]
    \item \textbf{Autenticación}: Registro/Login con email o Google
    \item \textbf{Exploración}: Búsqueda de tiendas y productos
    \item \textbf{Selección}: Agregado de productos al carrito
    \item \textbf{Configuración}: Dirección de entrega y método de pago
    \item \textbf{Pago}: Procesamiento seguro de la transacción
    \item \textbf{Seguimiento}: Monitoreo en tiempo real del pedido
    \item \textbf{Comunicación}: Chat con la tienda si es necesario
    \item \textbf{Recepción}: Confirmación de entrega
    \item \textbf{Evaluación}: Calificación del servicio
\end{enumerate}

\subsection{Flujo de Gestión de Tienda}
\begin{enumerate}[label=\arabic*.]
    \item \textbf{Configuración}: Setup inicial de la tienda
    \item \textbf{Inventario}: Gestión de productos y stock
    \item \textbf{Pedidos}: Recepción y procesamiento de órdenes
    \item \textbf{Preparación}: Actualización de estados de pedido
    \item \textbf{Comunicación}: Chat con clientes cuando necesario
    \item \textbf{Analytics}: Revisión de métricas y ventas
    \item \textbf{Notificaciones}: Alertas de inventario y pedidos
\end{enumerate}

\subsection{Flujo del Repartidor}
\begin{enumerate}[label=\arabic*.]
    \item \textbf{Disponibilidad}: Marcado como disponible
    \item \textbf{Asignación}: Recepción de pedidos asignados
    \item \textbf{Recolección}: Navegación a la tienda
    \item \textbf{Confirmación}: Pickup del pedido
    \item \textbf{Entrega}: Navegación al cliente con GPS
    \item \textbf{Comunicación}: Contacto con cliente vía WhatsApp
    \item \textbf{Finalización}: Confirmación de entrega
\end{enumerate}

\section{Base de Datos y Modelos}

\subsection{Colecciones de Firestore}
\begin{table}[H]
\centering
\begin{tabular}{@{}ll@{}}
\toprule
\textbf{Colección} & \textbf{Propósito} \\
\midrule
users & Información de usuarios y perfiles \\
stores & Datos de tiendas y configuración \\
products & Catálogo de productos \\
orders & Pedidos y su estado \\
payments & Transacciones de pago \\
inventory & Stock y movimientos \\
notifications & Historial de notificaciones \\
chat\_rooms & Conversaciones tienda-cliente \\
deliveries & Información de entregas \\
analytics & Métricas y estadísticas \\
\bottomrule
\end{tabular}
\caption{Estructura de base de datos en Firestore}
\end{table}

\subsection{Modelos de Datos Principales}
\begin{itemize}[itemsep=0.5em]
    \item \textbf{User Model}: Autenticación y perfil de usuario
    \item \textbf{Store Model}: Información y configuración de tiendas
    \item \textbf{Product Model}: Catálogo de productos con precios
    \item \textbf{Order Model}: Pedidos con estados y seguimiento
    \item \textbf{Payment Model}: Transacciones y métodos de pago
    \item \textbf{Inventory Model}: Stock y alertas de inventario
    \item \textbf{Analytics Model}: Métricas y KPIs
    \item \textbf{Notification Model}: Configuración de notificaciones
\end{itemize}

\section{Configuración y Despliegue}

\subsection{Requisitos del Sistema}
\begin{itemize}[itemsep=0.5em]
    \item \textbf{Flutter SDK}: 3.16.0 o superior
    \item \textbf{Dart SDK}: 3.2.0 o superior
    \item \textbf{Android Studio}: Para desarrollo Android
    \item \textbf{Xcode}: Para desarrollo iOS (macOS)
    \item \textbf{Firebase Project}: Configurado con todos los servicios
    \item \textbf{Google Maps API}: Claves de API configuradas
\end{itemize}

\subsection{Comandos de Construccion}
\begin{lstlisting}[language=bash, caption=Comandos principales]
# Instalar dependencias
flutter pub get

# Analizar codigo
flutter analyze

# Ejecutar tests
flutter test

# Limpiar proyecto
flutter clean

# Construir APK debug
flutter build apk --debug

# Construir APK release
flutter build apk --release

# Ejecutar aplicacion
flutter run
\end{lstlisting}

\subsection{Configuracion de Firebase}
\begin{enumerate}[label=\arabic*.]
    \item Crear proyecto en Firebase Console
    \item Habilitar Authentication (Email/Password y Google)
    \item Configurar Firestore con reglas de seguridad
    \item Activar Firebase Storage
    \item Configurar Firebase Messaging
    \item Descargar archivos de configuracion:
    \begin{itemize}
        \item \texttt{google-services.json} para Android
        \item \texttt{GoogleService-Info.plist} para iOS
    \end{itemize}
\end{enumerate}

\section{Analisis de Costos}

\subsection{Costos de Desarrollo}
\begin{table}[H]
\centering
\begin{tabular}{@{}lll@{}}
\toprule
\textbf{Recurso} & \textbf{Tiempo} & \textbf{Descripción} \\
\midrule
Análisis y Diseño & 40 horas & Planificación y wireframes \\
Desarrollo Frontend & 120 horas & UI/UX y funcionalidades \\
Integración Backend & 60 horas & Firebase y APIs \\
Testing y QA & 30 horas & Pruebas y correcciones \\
Documentación & 20 horas & Manual técnico y guías \\
\midrule
\textbf{Total} & \textbf{270 horas} & \textbf{Tiempo total de desarrollo} \\
\bottomrule
\end{tabular}
\caption{Estimación de tiempo de desarrollo}
\end{table}

\subsection{Costos de Infraestructura}
\begin{table}[H]
\centering
\begin{tabular}{@{}lll@{}}
\toprule
\textbf{Servicio} & \textbf{Costo Mensual} & \textbf{Límites} \\
\midrule
Firebase Firestore & \$0 - \$25 & Según uso y lecturas/escrituras \\
Firebase Auth & Gratis & Hasta 50,000 usuarios \\
Firebase Storage & \$0 - \$10 & Según almacenamiento usado \\
Google Maps API & \$0 - \$200 & Según consultas de mapas \\
Firebase Messaging & Gratis & Notificaciones ilimitadas \\
\midrule
\textbf{Total Estimado} & \textbf{\$0 - \$235} & \textbf{Para aplicación pequeña-mediana} \\
\bottomrule
\end{tabular}
\caption{Costos estimados de servicios en la nube}
\end{table}

\subsection{Costos de Distribución}
\begin{itemize}[itemsep=0.5em]
    \item \textbf{Google Play Store}: \$25 (pago único para cuenta de desarrollador)
    \item \textbf{Apple App Store}: \$99/año (membresía de desarrollador)
    \item \textbf{Certificados de Firma}: Incluidos en las membresías
\end{itemize}

\section{Referencias Tecnicas}

\subsection{Documentacion Oficial}
\begin{itemize}[itemsep=0.5em]
    \item Flutter Documentation: \url{https://docs.flutter.dev/}
    \item Firebase Documentation: \url{https://firebase.google.com/docs}
    \item Riverpod Documentation: \url{https://riverpod.dev/}
    \item Google Maps Flutter: \url{https://pub.dev/packages/google_maps_flutter}
    \item Material Design Guidelines: \url{https://material.io/design}
\end{itemize}

\subsection{Librerias y Paquetes}
\begin{itemize}[itemsep=0.5em]
    \item Pub.dev - Flutter Packages: \url{https://pub.dev/}
    \item Firebase Flutter Plugins: \url{https://firebase.flutter.dev/}
    \item FL Chart Documentation: \url{https://github.com/imaNNeo/fl_chart}
    \item Geolocator Plugin: \url{https://pub.dev/packages/geolocator}
\end{itemize}

\subsection{APIs y Servicios}
\begin{itemize}[itemsep=0.5em]
    \item Google Maps Platform: \url{https://developers.google.com/maps}
    \item Firebase Console: \url{https://console.firebase.google.com/}
    \item Google Cloud Console: \url{https://console.cloud.google.com/}
\end{itemize}

\section{Consideraciones de Seguridad}

\subsection{Autenticacion y Autorizacion}
\begin{itemize}[itemsep=0.5em]
    \item Tokens JWT manejados automaticamente por Firebase Auth
    \item Reglas de seguridad de Firestore por roles de usuario
    \item Validacion de permisos en el lado del servidor
    \item Sesiones seguras con renovacion automatica
\end{itemize}

\subsection{Proteccion de Datos}
\begin{itemize}[itemsep=0.5em]
    \item Encriptacion en transito (HTTPS/TLS)
    \item Encriptacion en reposo (Firebase)
    \item Validacion de datos de entrada
    \item Sanitizacion de datos sensibles
    \item Cumplimiento con mejores practicas de privacidad
\end{itemize}

\section{Mantenimiento y Escalabilidad}

\subsection{Monitoreo}
\begin{itemize}[itemsep=0.5em]
    \item Firebase Analytics para metricas de uso
    \item Crashlytics para reporte de errores
    \item Performance Monitoring para optimizacion
    \item Alertas automaticas de Firebase
\end{itemize}

\subsection{Escalabilidad}
\begin{itemize}[itemsep=0.5em]
    \item Arquitectura modular con Riverpod
    \item Firestore automaticamente escalable
    \item Caching estrategico en la aplicacion
    \item Optimizacion de consultas de base de datos
    \item Lazy loading de contenido
\end{itemize}

\section{Conclusiones}

El proyecto UberEats Clone representa una implementacion completa y robusta de un sistema de delivery de comida. La aplicacion cumple con todos los requisitos funcionales establecidos y proporciona una base solida para un producto comercial.

\subsection{Fortalezas del Proyecto}
\begin{itemize}[itemsep=0.5em]
    \item \textbf{Arquitectura Solida}: Separacion clara de responsabilidades
    \item \textbf{Tecnologias Modernas}: Stack tecnologico actualizado y mantenible
    \item \textbf{Escalabilidad}: Diseno preparado para crecimiento
    \item \textbf{Funcionalidad Completa}: Todas las caracteristicas de una app de delivery
    \item \textbf{Experiencia de Usuario}: Interfaces intuitivas y responsivas
    \item \textbf{Tiempo Real}: Actualizaciones instantaneas y seguimiento live
\end{itemize}

\subsection{Oportunidades de Mejora}
\begin{itemize}[itemsep=0.5em]
    \item Integracion con gateways de pago reales
    \item Implementacion de machine learning para recomendaciones
    \item Optimizacion adicional de rendimiento
    \item Expansion de funcionalidades de analisis
    \item Implementacion de tests automatizados mas extensivos
\end{itemize}

\vspace{2cm}

\begin{center}
\textcolor{primaryblue}{\rule{12cm}{0.5pt}}
\vspace{0.5cm}

\textit{Este manual técnico documenta completamente la implementación\\
del sistema UberEats Clone desarrollado con Flutter y Firebase.}

\vspace{0.5cm}
\textcolor{primaryblue}{\rule{12cm}{0.5pt}}
\end{center}

\end{document}